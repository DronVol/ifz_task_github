\section{Постановка задачи}

\par В безграничном объёме расплава на глубине около 30 км рассматривается одиночный пузырёк газа, наполненный смесью углекислого газа и водяного пара. Оба газа предполагаются совершенными (или с известной таблицей свойств). Подъём пузырьков предполагается значительно медленнее процессов массообмена с окружающим расплавом и скоростью движения границы пузырька, поэтому в модели рассматривается покоящийся единичный пузырёк. Также, изменениее температуры полагается пренебрежимо малым. Давление в окружающем расплаве предполагается постоянными, равным давлению насыщенного пара за вычетом $\Delta P\approx 1-10 MPa$.\\
\par В замкнутую систему уравнений входят следующие зависимости:\\
10) Уравнения диффузии для газов:\\
$$\frac{\partial c_{s}}{\partial t}+\nu_{r} \frac{\partial c_{s}}{\partial r}=\frac{1}{r^{2}}\frac{\partial}{\partial r}(D_{s}r^{2}\frac{\partial c_{s}}{\partial r});\; s=c, w;\;\;\;\nu_{r}=\frac{dR}{dt}$$\\
2) Уравнение баланса импульса на границе раздела расплав/газ:\\
$$P_{g}-P_{m}=\frac{2 \sigma}{R}+4\mu_{e}(\frac{1}{R}-\frac{R^{2}}{S^{3}})$$
$$S = (S_{0}^{3}-R_{0}^{3}+R^{3})^{1/3};\;\;\;\int\limits^{S}_{R}\frac{\mu(r,c_{w})}{r^{4}}dr$$\\
3) Уравнение баланса массы на границе раздела расплав/газ:\\
$$\frac{4\pi}{3}\frac{d}{dt}(R^3\rho_{g})=4\pi R^{2}\rho_{m}[D_{c}(\frac{\partial c_{c}}{\partial r})_{r=R}+D_{w}(\frac{\partial c_{w}}{\partial r})_{r=R}]$$
$$D_{i}=D_{i}(c_{w})\text{ --- известные функции};\;\;\;\rho_{m}=const$$\\
4) Уравнение состояния совершенного газа (или таблица для поиска связи давления с плотностью):\\
$$\frac{4\pi}{3}R^{3}=\rho_{g} \tilde{R}\frac{T}{P_{g}}$$