\section{Численная схема уравнений диффузии}

\par Сделаем замену координат в представленных уравнениях. Возьмём следующую замену: $$\d \xi = \frac{r-R}{S-R}$$
такую что $r = R$ $\xi = 0$, $r = S$ $\xi = 1$.
\par В уравнении 5 изменений нет, в уравнении 3 $$\tilde{\mu_{e}}=(S-R)\int\limits^{1}_{0}\frac{\mu(\xi, c_{w})}{(\xi(S-R)+R)^{4}}d\xi$$
\par В уравнении 4 правая часть принимает вид $$4\pi R\rho_{m}\left[D_{c}\left.\frac{\partial c_{c}}{\partial \xi}\right|_{\xi=0}+D_{w}\left.\frac{\partial c_{w}}{\partial \xi}\right|_{\xi=0}\right]$$
\par Для рассматриваемой замены координат имеем следующее:
$$\frac{\partial f}{\partial r}=\frac{\partial f}{\partial \xi}\frac{\partial \xi}{\partial r};\;\;\frac{\partial f(x,t)}{\partial t}=\frac{\partial f(\xi,t)}{\partial t}+\frac{\partial f(\xi,t)}{\partial \xi}\frac{\partial \xi}{\partial t}$$
\par Используя эти правила, получим:
$$\frac{\partial \xi}{\partial r}=\frac{1}{S-R};\;\;\frac{\partial \xi}{\partial t}=\frac{\dot{R}(\xi -1)-\dot{S}\xi}{S-R}$$

\par Для 2 уравнения, после подстановки решения $\nu_{r}$, имеем:\\
$$\d \frac{\partial c_{s}}{\partial t}+\frac{1}{r^{2}}\dot{R}R^{2}\frac{\partial c_{s}}{\partial r}=\frac{1}{r^{2}}\frac{\partial}{\partial r}\left(D_{s}r^{2}\frac{\partial c_{s}}{\partial r}\right)$$
Выполним вышеуказанную подстановку. После приведения членов, получим:
$$\frac{\partial c_{s}}{\partial t}+\frac{\partial c_{s}}{\partial \xi}\frac{1}{S-R}\left[\dot{R}(\xi-1)-\dot{S}\xi+\frac{\dot{R}R^{2}}{(\xi(S-R)+R)^{2}}\right]=\frac{1}{r^{2}}\frac{\partial}{\partial r}\left(D_{s}r^{2}\frac{\partial c_{s}}{\partial r}\right)$$
Для второго слагаемого введём обозначение для числителя выражения в квадратных скобках после приведения к общему знаминателю:
$$\kappa(\xi,t)=\left(\dot{R}(\xi-1)-\dot{S}\xi\right)\left(\xi^{2}(S-R)^{2}+R^{2}+2R\xi(S-R)\right)+\dot{R}R^{2}$$
Получаем
$$\frac{\partial c_{s}}{\partial t}+\frac{\partial c_{s}}{\partial \xi}\frac{1}{S-R}\frac{\kappa(\xi,t)}{r^{2}}=\frac{1}{r^{2}}\frac{1}{(S-R)^{2}}\frac{\partial}{\partial \xi}\left(D_{s}r^{2}\frac{\partial c_{s}}{\partial \xi}\right)$$
\par Проинтегрируем это уравнение по отрезкам от $\xi_{i}$ до $\xi_{j}$, умноженное на $r^{2}$. Проинтегрируем каждое слагаемое отдельно.
$$\int\limits^{\xi_{j}}_{\xi_{i}}\frac{\partial c_{s}}{\partial t}r^{2}d\xi=\frac{\partial c_{s}(\theta)}{\partial t}\left(\left(S-R\right)^{2}\frac{\xi^{3}}{3}+2\left(S-R\right)R\frac{\xi^{2}}{2}+R^{2}\xi\right)\;\;\;\text{(1)}$$
\par Для второго слагаемого отдельно найдём первообразную для $\kappa(\xi,t)$:\\
$$\int\limits^{\xi_{j}}_{\xi_{i}}\kappa(\xi,t)d\xi = \frac{\xi^{4}}{4}(S-R)^{2}(\dot{R}-\dot{S})+\frac{\xi^{3}}{3}[2R(S-R)(\dot{R}-\dot{S})-\dot{R}(S-R)^{2}]+$$
$$+\frac{\xi^{2}}{2}[(\dot{R}-\dot{S})R^{2}-2R\dot{R}(S-R)]=\hat{\kappa}(\xi,t)$$
\par Тогда для второго слагаемого имеем:\\
$$\int\limits^{\xi_{j}}_{\xi_{i}}\frac{\partial c_{s}}{\partial \xi}\frac{1}{S-R}\frac{\kappa(\xi,t)}{r^{2}}d\xi=\frac{\partial c_{s}(\theta)}{\partial \xi}\frac{\hat{\kappa}(\xi,t)}{S-R}\;\;\;(2)$$
\par Для тертьего слагаемого получаем:
$$\int\limits^{\xi_{j}}_{\xi_{i}}\frac{1}{r^{2}}\frac{1}{(S-R)^{2}}r^{2}\frac{\partial}{\partial \xi}\left(D_{s}(c_{w})r^{2}\frac{\partial c_{s}}{\partial \xi}\right)d\xi=\frac{1}{(S-R)^{2}}\left(D_{s}(c_{w})r^{2}\frac{\partial c_{s}}{\partial \xi}\right)\;\;\;(3)$$
\par Складывая эти части и вспоминая, что надо взять разности в двух точках, получим итоговое уравнение на промежутке $(\xi_{i},\xi_{j})$: 
$$\frac{\partial c_{s}(\theta)}{\partial t}\left.\left[\left(\left(S-R\right)^{2}\frac{\xi^{3}}{3}+2\left(S-R\right)R\frac{\xi^{2}}{2}+R^{2}\xi\right)\right]\right|_{\xi_{i}}^{\xi_{j}}+\frac{\partial c_{s}(\theta)}{\partial \xi}\left.\left[\frac{\hat{\kappa}(\xi,t)}{S-R}\right]\right|_{\xi_{i}}^{\xi_{j}}=$$
$$=\left.\left[\frac{1}{(S-R)^{2}}\left(D_{s}(c_{w})r^{2}\frac{\partial c_{s}}{\partial \xi}\right)\right]\right|_{\xi_{i}}^{\xi_{j}}$$


\par В уравнениях выше полагается, что в правой части стоит разница между значениями на концах промежутка интегрирования, функции от $\theta$ --- значения в промежуточной точке, которые можно вычислять как среднее от значений в крайних.