\section{Численная схема уравнений диффузии}

\par Сделаем замену координат в представленных уравнениях. Возьмём следующую замену: $$\d \xi = \frac{r-R}{S-R}$$
такую что $r = R$ $\xi = 0$, $r = S$ $\xi = 1$.
\par В уравнении 4 изменений нет, в уравнении 2 $$\tilde{\mu_{e}}=(S-R)\int\limits^{1}_{0}\frac{\mu(\xi, c_{w})}{(\xi(S-R)+R)^{4}}d\xi$$
\par В уравнении 3 правая часть принимает вид $$4\pi R\rho_{m}\left[D_{c}\left.\frac{\partial c_{c}}{\partial \xi}\right|_{\xi=0}+D_{w}\left.\frac{\partial c_{w}}{\partial \xi}\right|_{\xi=0}\right]$$
\par Для рассматриваемой замены координат имеем следующее:
$$\frac{\partial f}{\partial r}=\frac{\partial f}{\partial \xi}\frac{\partial \xi}{\partial r};\;\;\frac{\partial f(x,t)}{\partial t}=\frac{\partial f(\xi,t)}{\partial t}+\frac{\partial f(\xi,t)}{\partial \xi}\frac{\partial \xi}{\partial t}$$
\par Используя эти правила, получим:
$$\frac{\partial \xi}{\partial r}=\frac{1}{S-R};\;\;\frac{\partial \xi}{\partial t}=\frac{\dot{R}(\xi -1)-\dot{S}\xi}{S-R}$$

\par Для первого уравнения $$\d \frac{\partial c_{s}}{\partial t}+\nu_{r} \frac{\partial c_{s}}{\partial r}=\frac{1}{r^{2}}\frac{\partial}{\partial r}(D_{s}r^{2}\frac{\partial c_{s}}{\partial r})$$
выполним вышеуказанную подстановку. После приведения членов, получим:
$$\frac{\partial c_{s}}{\partial t}+\dot{R}\frac{\partial c_{s}}{\partial \xi}\left(\frac{1-\dot{R}(\xi -1)-\dot{S}\xi}{S-R}\right)=\frac{1}{(S-R)^{2}}D_{s}\left(2\frac{(S-R)}{\xi(S-R)+R}+\frac{\partial^{2} c_{s}}{\partial \xi^{2}}\right)$$
\par Проинтегрируем это уравнение по отрезкам от $\xi_{i}$ до $\xi_{j}$, умноженное на $\xi^{2}$. Проинтегрируем каждое слагаемое отдельно.
$$\int\limits^{\xi_{j}}_{\xi_{i}}\xi^{2}\frac{\partial c_{s}}{\partial t}d\xi=\frac{1}{3}\frac{\partial c_{s}}{\partial t}\xi^{3}\;\;\;\text{(1)}$$
$$\frac{\dot{R}}{S-R}\int\limits^{\xi_{j}}_{\xi_{i}}\xi^{2}\frac{\partial c_{s}}{\partial \xi}\left(\frac{1-\dot{R}(\xi-1)-\dot{S}\xi}{S-R}\right)d\xi=\frac{\dot{R}}{S-R}\left[\xi^{2}c_{s}(1-\right.$$
$$\left.-\dot{R}(\xi-1)-\dot{S}\xi)+(\dot{R}+\dot{S})c_{s}(\theta)\frac{\xi^{3}}{3}\right]\;\;\;\text{(2)}$$
$$\frac{D_{s}}{(S-R)^{2}}\int^{\xi_{j}}_{\xi_{i}}\left(2\frac{(S-R)}{\xi(S-R)+R}+\frac{\partial^{2} c_{s}}{\partial \xi^{2}}\right)\xi^{2}d\xi=\frac{D_{s}}{(S-R)^{2}}\left[2\frac{R^{2}ln\left((S-R)\xi+R\right)}{2(S-R)^{2}}+\right.$$
$$+\left.\frac{(S-R)\xi((S-R)\xi-2R)}{2(S-R)^{2}}+\frac{\partial c_{s}}{\partial \xi}\xi^{2}-\frac{\partial c_{s}(\theta)}{\partial \xi}\xi^{2}\right]\;\;\;\text{(3)}$$
\par В уравнениях выше полагается, что в правой части стоит разница между значениями на концах промежутка интегрирования, функции от $\theta$ --- значения в промежуточной точке, которые можно вычислять как среднее от значений в крайних.
