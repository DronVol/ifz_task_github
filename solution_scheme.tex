\section{Схема решения}
\par После того, как заданы внешние зависимости, граничные и начальные условия, система уравнений решается следующим образом. Делается шаг по времени, предполагается некоторое значение $R$. Далее решается уравнение 2, из которого становятся известными концентрации газов в расплаве. Из уравнения 4 получаем плотность $\rho_{g}$. Теперь из уравнений 3 и 5 независимо получаются давления $p_{g}$. Если разница между полученными давлениями меньше, чем некторое значение ошибки, то делается новая иттерация. Когда достигается приемлимая разница, можно сделать следующий шаг по времени. 